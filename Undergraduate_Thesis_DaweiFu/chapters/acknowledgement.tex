\chapter{致谢}

以这篇致谢作为对我本科四年的总结。我自己是通过竞赛降分进入的北大,高考也没有同过北大的裸分线,所以一开始进北大也不在物理学院,而是被调剂到了工学院。因为竞赛的失利,所以高三末期到刚进北大,也没有特别大的愿望要来到物理学院。

大一在工学院的时候,一开始也没有特别宏大的目标,只是正常地学习生活玩耍,同时在因为工学院的部分专业要求,我在大一还修了数学分析和高等代数,在大一并不算忙碌的时间里,这两门纯粹的数学课让我遨游在思考数学的海洋里,而不用关心现实中繁杂的事务,至今回想起来仍然觉得十分美好,并且特别感谢教我数学分析的史一鹏老师,他是一位十分善良和蔼关心学生的老师,为我在北大碰到的教师群体定下了主色调。同时,由于工学院还要修适当的物理课,我在旁听了一门工学院开的物理课后,觉得还是理科院系开的物理课更对我的胃口,便决定选了物理学院的力学课,这门课的老师是孟策老师,他是一位特别爱笑、心理年龄十分年轻的中年教师,至今我仍然常常回忆起他坐在转椅上拿着杠铃,一边自我旋转一边向我们讲授角动量的概念,让人忍俊不禁。

到了大一的下学期,我依然没有强烈的要去哪个方向学习/转专业的想法,只是忽然看见转专业通知,便又想起了我的老本行:物理。“要不要到物理学院去看看呢?可以先报个名,之后再慢慢做决定吧”,这就是我当时的想法,于是我顺利通过了转专业的笔试和面试,记得面试我的是王稼军奶奶,刘树新老师(留着哈利波特中斯内普教授一样的发型)以及实验中心的张朝晖老师。他们问我以后到物理学院想来学什么方向,做实验还是理论,我说我觉得实验理论都可以吧(但其实那时候我知道我实验动手能力很差),方向还没有想好,但是想做和量子有关的东西(因为那时候看新闻上的量子计算量子通信特别火热)。刘树新老师笑了笑:“物理什么东西和量子无关呀?”

于是,对于我这种选择困难症来说,最后也就在大二开学时顺其自然地来到了物理学院,和我一起从工学院转来的有谭奕和许安冬,二者后来都和我成为了很好的朋友。这里补充一个有趣的事,大一下时我知道自己报名了转专业可能转到物院,于是就想着要开始选量子力学(四大我第一门学的课),因为知道自己是平转所以比同级的同学少了一年在物院的学习,要抓紧赶上才能尽快投入本研。可惜运气不好的是正好碰上了刘玉鑫老师开的量子力学,刘老师刀子嘴豆腐心,对学生十分关心尊重,但他上课起来却毫不留情,巨大的作业量和凶残的小测让我很快就感觉自己没有跟上进度,只得第一次动用期中退课的选项。

到了大二上的时候,我再次选了量子力学,这次碰上了年纪稍大但是为人幽默风趣,操着一口京腔官话(并不是现在的北京话)的田光善老师,他制作了详细精美的量子力学笔记,同时辅以适当的作业,让我感觉课程难度放缓了许多,同时也激发了我认真学习的兴趣,这门考试我最后拿了97分,在我四年的成绩单中也算得上是一个漂亮的数字,大大地鼓舞了我学物理的兴趣。在课程结束后,寒假期间,我去找了这位幽默可爱的田老师聊天(聊科研聊人生),并且尝试问他是否还招本科生做本研,因为我个人很喜欢他的幽默特质,具有很强的人格魅力。可惜田老师说,他2018年就已经退休了,自此不再带学生了。田老师自己是做凝聚态理论的,我至今还在常常想:如果田老师当时年轻几岁,还带学生,也许我就会一直跟着他做本研甚至读博士了,那我就应该从此钻研与凝聚态理论,而与现在的人生轨迹截然不同了。当时,我在叹息之余还是和他聊了许多和物理有关无关的事,那次谈话至今想起来都是本科生涯最美好的回忆之一。

和田老师谈完之后,我回到湖北家中,这时候正好爆发了新冠疫情,我回家的那两天正好就是官方宣布不明原因肺炎为新冠肺炎,并且开始动用全国力量支援武汉疫情。回到家中一周后,武汉开始封城,同时过了几天我所在的荆州市也开始封城,于是我就在家中开始了我的大二下学期,我知道我到了该进入本研的时间节点,于是在家中也开始疯狂联系各个方向老师,包括做凝聚态实验的老师,以及做高能实验的高原宁院长、冒亚军教授等,最后因为要远程开展科研的原因,我便进入了做高能实验分析的冒亚军老师组,做了大约一年的BESIII实验相关课题,在这个过程中感谢全程带我入门的宋昀轩师兄,为人无私热情,技术高超。

同时,在大三回到校园后,我又在办公室认识了本科期间重要的一位朋友——李聪乔师兄,当时他是隔壁李强老师组做CMS实验的,和我一见如故,常常约饭聊天。在BSEIII上的课题做了很久,有了一定的进展,但我卡在课题上的一个地方卡了很久,由于只有我一个人在推动这个课题,所以对我来说是个很巨大的挑战。之后由于一些原因,我个人就逐渐丧失了对BES实验的热情,正好这个时候到了夏令营保研的阶段,我想试试换个方向,于是李聪乔师兄就把我推荐到了李强老师那边。而正好我的导师冒亚军老师和李强老师有合作,我便在冒老师名下跟着李强老师从大三暑假开始了我在CMS实验的探索。

在这个探索过程中,我也就做出了本篇毕业论文的工作,这个工作成功地和UCSD的Javier组建立了合作关系,并且在CMS官方会议上给予了报告,作为博士生涯的铺垫来说是一个不错的开端,也为以后更多的合作奠定了基础。

于是,在这篇论文的最后,我要感谢全程指导我科研工作的冒亚军老师、李强老师,为我科研工作牵线搭桥不吝赐教的李聪乔师兄,以及两年多来在办公室与我一起奋战在科研一线并且帮助过我的宋昀轩师兄、郭启隆师兄、赵宇哲师兄、邓森师兄、钱思天师兄、李彦谷师兄、谢昕海师兄、相腾师兄和王轩师姐(以上排名不分先后),以及给了我不少人生建议的安莹师姐。

回首四年,我已与刚踏进北大时的自己发生了很多天翻地覆的改变,比起刚进来时的随波逐流,我现在更感到自己有做出成就改变世界的可能性,并且愿意为此奉献汗水和努力。希望我自己在另一个四年后,能够交出一份让现在的我满意的人生答卷,与世界共勉。